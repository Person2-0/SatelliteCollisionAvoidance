\documentclass{beamer}
\usetheme{PaloAlto}

% Title page details: 
\title{Satellite Collision Avoidance} 
\author{Gabriella Armijo}
\institute{Institute for Computing in Research}
\date{\today}
\logo{\large \LaTeX{}}


\begin{document}

% Title page frame
\begin{frame}
    \titlepage 
\end{frame}

% Remove logo from the next slides
\logo{}


% Outline frame
\begin{frame}{Outline}
    \tableofcontents
\end{frame}

% slide 3
\section{Project Goals}
\begin{frame}{Project Goals}
\begin{itemize}
\item Track satellites to see how likely they were collide. 
\item See how often they got within 100 km of each other. 
\item Analyze those results to see what satellites showed up the most.
\end{itemize}
\end{frame}

%slide 4
\section{Introduction}
\begin{frame}{Kessler Syndrome}
A scenario in which the density of objects in Low Earth Orbit (LEO) is high enough that each collision creates debris that increases the likelihood of more collisions.
\end{frame}

% slide 5
\section{Methods}
\begin{frame}{3D Plot}
\centering
\includegraphics[width=0.8\textwidth]{Screenshot from 2022-08-04 10-29-10.png}
\end{frame}

%slide 6
\begin{frame}{Conjunction Plots}
\includegraphics[width=5cm]{Screenshot from 2022-08-04 10-32-50.png}
\includegraphics[width=5cm]{Screenshot from 2022-08-04 10-33-47.png}
\end{frame}

%slide 7
\begin{frame}{Satellite Point of View}
\centering
\includegraphics[scale=0.25]{SatPOV.png}
\end{frame}

% \begin{frame}{Starlink}
%add the starlink, fengyun, cosmos and other conjunction plots.
%talk about what starlink does when they launch satellites
%talk about what the russians did to the cosmos 1408 satellite
%talk about what the chinese satellite
%\end{frame}

% slide 8 
\begin{frame}{Dot Product}
\begin{figure}
\includegraphics[scale=0.5]{Screenshot from 2022-08-02 18-44-33.png}
	\caption{Dot Product}
\end{figure}
\end{frame}

% \begin{frame}
% \begin{figure}
% \includegraphics[scale=0.5]{Screenshot from 2022-08-02 18-44-08}
% 	\caption{Cross Product}
% \end{figure}
% \end{frame}

%slide 9
\begin{frame}
\includegraphics[scale=0.4]{Screenshot from 2022-08-02 18-42-48.png}
\end{frame}

%slide 10
\section{Results}
\begin{frame}{Results}
\includegraphics[scale=0.4]{Screenshot from 2022-08-02 19-39-49.png}
\includegraphics[scale=0.35]{Screenshot from 2022-08-02 19-40-35.png}
\includegraphics[scale=0.35]{Screenshot from 2022-08-02 19-40-58.png}
\includegraphics[scale=0.4]{Screenshot from 2022-08-03 09-18-57.png}
\includegraphics[scale=0.35]{Screenshot from 2022-08-03 09-19-42.png}
\includegraphics[scale=0.4]{Screenshot from 2022-08-03 09-20-10.png}
\end{frame}

%slide 11
\section{Relevance}
\begin{frame}{Why is this important?}
\begin{itemize}
\item Prevents Collisions
\item Keeping tabs on growing constellations
\item Understanding satellite movement
\end{itemize}
\end{frame}

%slide 12
\section{Future Work}
\begin{frame}{Future Work}
\begin{itemize}
\item Conjunction Plots
\item Starlink orbital readjustments
\item Future Collisions
\end{itemize}
\end{frame}

% Acknowledgments
\begin{frame}{Acknowledgments}
I would like to thank my mentor, David Palmer, for everything he has taught me.
\vfill
I would also like to thank the Institute for Computing in Research and everyone involved for giving me and my fellow interns this opportunity.
\end{frame}

% references
\section{References}
\begin{frame}{References}
\begin{itemize}
\item [1] Mann, A., Pultarova, T.,  Howell, E. (2022, April 14). SpaceX Starlink Internet: Costs, Collision Risks and How it Works. Available at https://www.space.com/spacex-starlink-satellites.html
\item [2] McKnight, D., Shouppe, M., (2021, November 18). Analysis of the Cosmos 1408 Breakup Available at https://leolabs-space.medium.com/analysis-of-the-cosmos-1408-breakup-71b32de5641f
\item [3] Lambert, J. (2018, September). Fengyun-1C Debris Cloud Evolution Over One Decade. % Available at https://ui.adsabs.harvard.edu/abs/2018amos.confE..50L/abstract#:~:text=Over%20a%20decade%20ago%2C%20on,an%20inclination%20of%2098.8%20degrees.
\item [4]  Wall, M. (2018, November 15,). Kessler Syndrome and the Space Debris Problem Available at https://www.space.com/kessler-syndrome-space-debris
\item [5] Weeden, B. (2010 November 10,) 2009 Iridium-Cosmos Collision Fact Sheet %https://swfound.org/media/205392/swf_iridium_cosmos_collision_fact_sheet_updated_2012.pdf
\end{itemize}
\end{frame}

\end{document}
